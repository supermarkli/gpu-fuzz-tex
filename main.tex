\documentclass[letterpaper,twocolumn,10pt,review,anonymous]{article}

\usepackage{usenix}
%%
%% \BibTeX command to typeset BibTeX logo in the docs
\AtBeginDocument{%
  \providecommand\BibTeX{{%
    Bib\TeX}}}

\newcommand{\outline}[1]{\mytodocyan{[outline: #1]}}
\newcommand{\mytodocyan}[1]{\textcolor{cyan}{\ding{46}~{\sf}~#1}}

\newcommand{\hy}[1]{\mytodopink{[hy: #1]}}
\newcommand{\mytodopink}[1]{\textcolor{violet}{\ding{46}~{\sf}~#1}}

\newcommand{\fw}[1]{\mytodored{[fw: #1]}}
\newcommand{\mytodored}[1]{\textcolor{red}{\ding{46}~{\sf}~#1}}

\newcommand{\yj}[1]{\mytodopink{[yj: #1]}}

\definecolor{wsorange}{RGB}{245,166,115}
\newcommand{\mytodoblue}[1]{\textcolor{blue}{\ding{46}~{\sf}~#1}}
\newcommand{\mytodoorange}[1]{\textcolor{wsorange}{\ding{46}~{\sf}~#1}}
\newcommand{\ws}[1]{\mytodoorange{[ws: #1]}}
\newcommand{\zl}[1]{\textcolor{SeaGreen}{\ding{46}~[zl: #1]}}
\newcommand{\fixme}[1]{{\color{blue}{#1}}} 
\newcommand{\sw}[1]{{\color{blue}{\small SW: #1}}} 

\newcommand{\bug}[1]{\noindent\textit{#1}}

\newcommand{\parh}[1]{\noindent\textbf{#1}}
\newcommand{\parhs}[1]{\noindent\underline{\textit{#1}}}
\usepackage{threeparttable}
\usepackage{xspace}
\usepackage{amsmath,amsfonts}
\usepackage{graphicx}
\usepackage{textcomp}
\usepackage{xcolor}
\usepackage{threeparttable}
\usepackage{colortbl}
\usepackage{graphicx}
\usepackage{listings}
\usepackage{color}
%\usepackage[dvipsnames]{xcolor}
\usepackage{array}
\usepackage{float}
\usepackage{graphicx}
\usepackage{multirow}
\usepackage{colortbl}
\usepackage[ruled,linesnumbered,boxed]{algorithm2e}
\usepackage{algpseudocode}
\usepackage{pifont}
\usepackage{enumitem}
\usepackage{balance}
\usepackage{xurl}
\usepackage{hyperref}
\usepackage[T1]{fontenc}
\usepackage[utf8]{inputenc}
\usepackage{caption}
\usepackage{booktabs}
\usepackage{tabularx}
\usepackage{diagbox}
\usepackage{amsmath}
\usepackage{listings}
\usepackage{makecell}
\usepackage{verbatim}
\usepackage{acronym}
\usepackage[super]{nth}
\usepackage{tikz}
\usepackage{amsmath}
\usepackage{filecontents}
\usepackage[misc]{ifsym}
\usepackage{grffile}
\usepackage{tikz,pgf}
\usepackage{amsthm}
\usepackage[most]{tcolorbox}
\usepackage{enumitem}
\usepackage{trimclip}

\newtheorem*{mydef}{Definition}
\newtheorem*{mycol}{Corollary}

\usetikzlibrary{calc}
\usetikzlibrary{positioning, fit, backgrounds, arrows.meta}
\makeatletter
\def\mfontsize{\f@size}
\newcommand*\circled[1]{\tikz[baseline=(char.base)]{
		\node[shape=circle,draw,inner sep=0pt] (char) {#1};}}


\newcommand{\F}{Fig.}
\newcommand{\E}{Eq.}
%\renewcommand{\F}{Figure}
\newcommand{\T}{Table}
%\renewcommand{\S}{Section}
\renewcommand{\S}{Sec.}
\newcommand{\A}{Alg.}
\newcommand{\highlight}[1]{{\color{red}\textbf{#1}}}



\usepackage{xcolor}

%\usepackage{minted}  % Commented out: not used and requires pygmentize

\definecolor{codegreen}{rgb}{0,0.6,0}
\definecolor{codegray}{rgb}{0.5,0.5,0.5}
\definecolor{codepurple}{rgb}{0.58,0,0.82}
\definecolor{backcolour}{rgb}{0.95,0.95,0.92}
\definecolor{cadmiumgreen}{rgb}{0.0, 0.42, 0.24}
\lstdefinestyle{mystyle}{
	%keywordstyle=\color{cadmiumgreen},
	escapeinside={(*@}{@*)},
	backgroundcolor=\color{backcolour},   
	commentstyle=\color{codegreen},
	keywordstyle=\color{magenta},
	numberstyle=\tiny\color{codegray},
	stringstyle=\color{codepurple},
	frame=shadowbox,
	basicstyle=\ttfamily\footnotesize,
	breakatwhitespace=false,         
	breaklines=true,                 
	captionpos=b,                    
	keepspaces=true,                 
	numbers=left,                    
	numbersep=5pt,                  
	showspaces=false,                
	showstringspaces=false,
	showtabs=false,                  
	tabsize=2,
}

\lstdefinestyle{gdb}
{
	backgroundcolor=\color{backcolour},
	%backgroundcolor=\color{black},
	%keywordstyle=\color{cadmiumgreen},
	keywordstyle=\color{magenta},
	escapeinside={(*@}{@*)},
	basicstyle=\scriptsize\color{black}\ttfamily,
	frame=shadowbox
}

%\renewcommand{\baselinestretch}{0.986}


\usepackage{subcaption}
\begin{document}

\title{\tool: Capturing Memory Corruption on NVIDIA GPUs}

\author{Anonymous Authors}


\maketitle
\begin{abstract}
  Modern GPU applications, particularly in machine learning and scientific
  computing, are increasingly affected by memory corruption bugs due to their
  reliance on memory-unsafe languages like C/C++. However, existing solutions
  either depend on hardware/software that is not available on commodity GPUs,
  or incur prohibitive performance overheads, rendering them impractical for
  real-world deployment.

  We present \tool, a novel GPU sanitizer that is readily deployable on
  commodity NVIDIA GPUs. \tool employs a hybrid metadata scheme combining
  pointer tagging with in-band buffer bounds to enable accurate and efficient
  memory safety validation. \tool also introduces mechanisms such as stack epoch
  tracking and virtual address randomization to mitigate metadata confusion
  caused by temporal corruption.

  Our security evaluation on 33 programs demonstrates that \tool uniquely
  achieves comprehensive coverages of both spatial and temporal bugs among
  existing GPU sanitizers. Moreover, our performance benchmarks on 44 programs,
  including large-language models like LLaMA2-7B and LLaMA3-8B, show
  that \tool incurs an average slowdown of 13\% and a negligible
  memory overhead of 0.3\%. 

\end{abstract}

\section{Introduction}
\label{sec:intro}

GPUs have become the cornerstone of modern computing, especially in the realm of artificial intelligence. Deep learning frameworks such as PyTorch, TensorFlow, and PaddlePaddle, which are the core software layer built upon GPUs, directly impact the reliability of countless applications across various domains. The GPU backends of these frameworks are extraordinarily complex, comprising a vast amount of code from low-level libraries like cuDNN and cuBLAS, as well as numerous handwritten CUDA kernels. This complexity makes them a fertile ground for security vulnerabilities and stability issues, where even a minor computation error or memory out-of-bounds access can lead to severe consequences.

However, traditional fuzzing methodologies prove ineffective in this context. The random data streams they generate are almost invariably rejected by the frameworks' stringent input validation checks on tensor shapes and data types. As a result, these fuzzers fail to reach the deeper computational logic within the GPU kernels, leaving a critical attack surface untested.

Our key insight is that to effectively test the GPU backend, the fuzzer must understand and respect the semantic constraints of the operators. For example, the relationship between input and output shapes in a convolutional layer is mathematically defined; generating inputs that adhere to these rules is crucial for bypassing frontend validation and stress-testing the underlying CUDA kernels.

Based on this insight, we have designed and implemented \textbf{GPU-Fuzz}, a novel, cross-framework fuzzing system. GPU-Fuzz automates the entire closed-loop process of constraint modeling, solving, test case generation, cross-framework execution, and crash reproduction. It systematically translates the semantic rules of deep learning operators into formal constraints, uses the Z3 SMT solver to generate valid and boundary-aware inputs, and then executes these tests on the GPU backends of major deep learning frameworks.

The main contributions of this paper are as follows:
\begin{itemize}
    \item We propose and implement GPU-Fuzz, the first system to apply Z3 constraint solving to the problem of cross-framework GPU fuzzing for deep learning applications.
    \item We demonstrate the effectiveness of our approach by discovering a significant number of previously unknown bugs in widely-used frameworks like PyTorch, TensorFlow, and PaddlePaddle. 
    \item We provide a comprehensive, open-source dataset of real-world bugs discovered by our tool, complete with detailed logs and reproducer scripts, which can serve as a valuable resource for developers and security researchers.
    \item We argue that our constraint-solving approach is significantly more efficient than traditional random fuzzing by ensuring a high ratio of valid test cases that reach the GPU backend.
\end{itemize}

\section{Background and Prior Work}
\label{sec:bg}

\subsection{GPU Architecture}

\F~\ref{fig:gpu_mem} illustrates the architecture of a typical GPU. Streaming
Multiprocessors~(SMs) are the basic compute units of a GPU. Each SM contains
multiple computing cores to execute threads in parallel. A GPU can have tens to
hundreds of SMs, enabling it to execute thousands of threads simultaneously.

\begin{figure}[htbp]
	\centering
	\includegraphics[width=0.7\linewidth]{figs/gpu_mem.pdf}
	\caption{GPU memory architecture.}
	\label{fig:gpu_mem}
\end{figure}


To achieve low-latency memory access, GPUs employ different types of memory to
meet diverse latency and bandwidth requirements. As shown in
\F~\ref{fig:gpu_mem}, the GPU memory system comprises a hierarchy of shared,
local, and global memory.

\parh{Shared memory} is a small but fast on-chip memory region shared by all
threads within the same SM. Unlike off-chip memory, shared memory can be
accessed with low latency; it is 100$\times$ faster than off-chip
memory~\cite{sharedmem}. This speed comes at the cost of limited capacity, for
example, 64\,KB per SM. Because of its low latency, the shared memory is often
used to store data that requires frequent access to avoid accessing costly
off-chip memory. Despite its name, shared memory is a scratchpad; some GPUs
allow configuring the ratio between shared memory and L1 cache, but the total
size is still limited.

\parh{Local memory} is an area within off-chip memory that is private to each
thread. It stores the thread's stack and is therefore referred to as stack
memory. Similar to the stacks on CPUs, local memory is not persistent and is
freed once all the threads within the same function~(i.e., kernel) exit.
The allocation of local memory is implicitly managed by NVIDIA's closed-source
driver, meaning it cannot be directly controlled from a low-level
perspective~(e.g., changing its address mapping).

\parh{Global memory} also resides in the same off-chip memory as local memory,
but can be managed from the CPU side. Global memory is allocated through
runtime APIs~(e.g., \texttt{cudaMalloc}). Unlike local memory, which is freed
once its threads exit, global memory is persistent and accessible by all
threads across all GPU functions until it is explicitly freed~(e.g.,
\texttt{cudaFree}). Additionally, NVIDIA partially open-sources the kernel
driver~\cite{opennv}; its exposed APIs allow developers to customize the
allocation of global memory. This capability enables \tool to implement an
MMU-based pointer tagging scheme, which will be discussed in
\S~\ref{subsec:tag}.

\parh{Cache hierarchy.}~As illustrated in \F~\ref{fig:gpu_mem}, the GPU has a
cache hierarchy to reduce the latency of off-chip memory. Each SM has a private
L1 cache and a shared L2 cache. Furthermore, to reduce the latency of page
table translation, each SM also has a private L1/L2 TLB and a shared L3 TLB.
This hierarchical design reduces the latency of memory access from the
numerous threads. However, due to its highly parallel nature, a GPU
application must be carefully designed~(e.g., avoiding accesses to pointers
that are widely dispersed across memory) to prevent TLB thrashing. In our
observation, retrieving sanitizer metadata from a shadow memory region as in
cuCatch~\cite{tarek2023cucatch} and AddressSanitizer~(ASAN)~\cite{asan} is
likely to cause this issue. To address this, \tool embeds metadata in-band at
the beginning of each buffer, a design choice well-suited to the GPU's cache
architecture.


\subsection{Pointer Tagging}
\label{subsec:tag}
\tool relies on pointer tagging to retrieve the metadata~(see
\S~\ref{subsec:overview}); we discuss its implementation on GPUs.

\parh{MMU-based global pointer tagging.}~Though major CPU vendors such as Intel
have announced their pointer tagging features~(e.g., Intel LAM~\cite{lam}), no
equivalent features exist on GPU. Fortunately, NVIDIA's open-source driver
exposes interfaces for managing the allocation of global memory, which allow
manipulation of virtual addresses for global buffers via the GPU MMU. We
thereby leverage the MMU to customize the upper bits of the virtual
address~(VA) to encode metadata, such as tags.

\begin{figure}[htbp]
	\centering
	\includegraphics[width=0.8\linewidth]{figs/gpu_mmu.pdf}
	\caption{GPU MMU page table format.}
	\label{fig:gpu_mmu}
\end{figure}

For instance, \F~\ref{fig:gpu_mmu} shows the page table format of
Blackwell\footnote{NVIDIA's GPUs have different page table designs across
generations; we provide this example for illustration.} GPUs~\cite{blackwell}.
As shown in the figure, a modern GPU supports multiple page sizes, including
\ding{192}\,4\,KB, \ding{193}\,64\,KB and \ding{194}\,2\,MB. Translation for
all these page sizes follows the same traversal process, driven by bits 56
through 21 of the VA. The page directory 5~(PD5) has only two entries, which
store the addresses of PD4 tables and are indexed by bit 56 of the VA. Page
directories PD4 through PD1 each contains 512 entries, indexed by nine bits of
the VA. PD0 differs, containing only 256 entries indexed by eight bits of the
VA. Each entry in PD0 can either be viewed as one 16-byte entry or two 8-byte
entries. In the former case, the 16-byte entry directly stores the address of a
2\,MB page frame~(\ding{192} in \F~\ref{fig:gpu_mmu}). In the latter case, the
lower half 8-byte entry points to a 64\,KB page frame~(\ding{193} in
\F~\ref{fig:gpu_mmu}) and the upper half 8-byte entry points to a 4\,KB page
frame~(\ding{194} in \F~\ref{fig:gpu_mmu}).

These PDx tables govern the translation result of a VA; therefore, by
configuring the page table, we can embed metadata within the higher bits of the
VA, effectively making these bits part of the pointer's tag. For example, \tool
enforces all objects of 256\,B to be allocated in the VA region with bits
[46:41]$=log(256)-log(16)+1=6$, with 16\,B being the minimal allocation size.
This approach enables flexible tagging of global buffer pointers through
judicious page table configuration.

\parh{Local/shared pointer tagging.}~For objects allocated in the local/shared
memory, their virtual addresses cannot be arbitrarily manipulated for tagging,
unlike those in global memory. This limitation arises because these addresses
are exclusively managed by NVIDIA's closed-source runtime. Instead, we adopt a
pseudo-pointer based approach. \F~\ref{fig:local_tag} presents a code snippet
that illustrates the tagging process of local memory pointers. In this snippet,
we allocate a local memory buffer of 24 bytes with 32-byte alignment~(line 1).
Subsequently, at line 2, a pseudo-pointer \texttt{\%tagged} is constructed by
embedding the necessary metadata~(e.g., its $2^n$ aligned size) into the
original pointer's value. As \texttt{\%tagged} is not a directly dereferencable
memory address, the tagging procedure must be reversed at line 4 to recover the
original, valid pointer \texttt{\%raw}. This raw pointer is then used to access
the memory buffer at line 5. This technique allows local/shared memory buffers
to be effectively tagged, albeit with a slight computational overhead due to
the additional arithmetic operations.

\begin{figure}[htbp]
	\centering
	\includegraphics[width=0.7\linewidth]{figs/local_tag.pdf}
	\caption{Local/shared memory tagging.}
	\label{fig:local_tag}
\end{figure}

\noindent\underline{Compatibility.}~A potential compatibility arises if a
``pseudo-pointer'' is passed to an external function, as the function would be
unable to dereference it directly. While such scenarios are infrequent in
typical CUDA applications (specifically, passing a local/shared pointer to an
external function), \tool mitigates this by automatically untagging such
pointers before they are passed to external functions.

\newcommand*\emptycirc[1][.8ex]{\tikz\draw (0,0) circle (#1);}
\newcommand*\halfcirc[1][.8ex]{%
	\begin{tikzpicture}
		\draw[fill] (0,0)-- (90:#1) arc (90:270:#1) -- cycle ;
		\draw (0,0) circle (#1);
	\end{tikzpicture}}
\newcommand*\fullcirc[1][.8ex]{\tikz\fill (0,0) circle (#1);}

\begin{table*}[htbp]
	\centering
	\caption{Comparison between previous GPU memory safety solutions and \tool.}
	\label{tbl:sum}
	\resizebox{.9\linewidth}{!}{%
		\begin{tabular}{lccccccc}
			\hline
			\textbf{Name}                             &
			\textbf{Base}                             &
			\textbf{Mechanism}                        &
			\textbf{Spatial}                          &
			\textbf{Temporal}                         &
			\textbf{Deployability}                    &
			\textbf{Perf. Overhead}                   &
			\textbf{Mem. Overhead}                                                                                                                      \\ \hline
			\textbf{GPUShield}~\cite{lee2022shield}   & HW & Tagging  & \fullcirc & \emptycirc & \ding{56} & 1\%           & None                       \\ \hline
			\textbf{IMT}~\cite{imt}                   & HW & Tagging  & \halfcirc & \emptycirc & \ding{56} & 4\%           & None                       \\ \hline
			\textbf{LMI}~\cite{lee2025lmi}            & HW & Aligning & \halfcirc & \halfcirc  & \ding{56} & 0.2\%         & $2^n$-fragmentation        \\ \hline
			\textbf{GMOD}~\cite{di2018gmod}           & SW & Redzone  & \halfcirc & \emptycirc & \ding{52} & 2.9\%         & 12\,B/buf                  \\ \hline
			\textbf{clArmor}~\cite{clarm}             & SW & Redzone  & \halfcirc & \emptycirc & \ding{52} & 9.6\%         & 4\,B/buf                   \\ \hline
			\textbf{compute-sanitizer}~\cite{compsan} & SW & Redzone  & \halfcirc & \halfcirc  & \ding{52} & 1,400\%       & Unknown                    \\ \hline
			\textbf{cuCatch}~\cite{tarek2023cucatch}  & SW & Tagging  & \halfcirc & \halfcirc  & \ding{56} & 19\%          & 160\,M + 12.5\%            \\ \midrule[1.5pt]
			\textbf{\tool}                            & SW & Hybrid   & \fullcirc & \fullcirc  & \ding{52} & \textbf{13\%} & \textbf{16.5\,M + 8\,B/buf} \\ \hline
		\end{tabular}%
	}
\end{table*}


\subsection{Memory Safety Solutions on GPUs}
\label{subsec:first_rw}

Memory safety is a long-standing problem. This section reviews various
memory safety solutions designed for GPUs. \T~\ref{tbl:sum} summarizes these
approaches. Though some tools (e.g., GPUShield) might not be formally named
``sanitizers'', we categorize them as such for simplicity, given their common
function of detecting memory corruption.

\parh{Redzone-based.}~Redzone is a widely used technique to detect memory
corruption. A redzone-based sanitizer typically allocates a shadow memory
region to track the status of the main memory and checks if the accessed memory
has been flagged as ``redzone'' during runtime. An example of a redzone-based
sanitizer is NVIDIA's compute-sanitizer~\cite{compsan}. It relies on binary
rewriting to insert checks before each memory access, and thus introduces
a considerable performance penalty due to the additional instructions inserted.
As shown in \T~\ref{tbl:sum}, our evaluation shows that compute-sanitizer leads
to an average slowdown of 15$\times$. Furthermore, as this approach primarily
places redzones between buffers, it cannot detect overflows that occur from one
buffer into an adjacent, valid buffer (e.g., an overflowed access from buffer A
into buffer B).

Other than compute-sanitizer, there are also GPU sanitizers like GMOD and
clArmor~\cite{di2018gmod,clarm} that are canary-based. By inserting a guard
value at the end of each memory allocation, these sanitizers detect memory
corruption by checking if the guard value is modified. As \T~\ref{tbl:sum}
shows, though this simplified approach significantly reduces the performance
overhead compared to compute-sanitizer, it also loses the capability to detect
temporal memory corruption~(e.g., use-after-free) and provides even worse
detection accuracy; any overflow that skips the guard value is not detected.

\parh{Tagging-based.}~Tagging-based sanitizers attach a tag to each pointer.
When memory access occurs, the sanitizer retrieves the metadata associated
with the tag and checks if the access is valid. For instance, NVIDIA's
cuCatch~\cite{tarek2023cucatch} embeds a tag into each pointer and queries a
two-level structure for the valid bounds of the buffer. Built upon the compiler's
backend, cuCatch shows superior performance~(19\% overhead as shown in \T~\ref{tbl:sum})
compared to approaches that rely on binary rewriting~(e.g., compute-sanitizer).
However, relying on the backend restricts it from obtaining accurate bounds,
as these are only available on the frontend; this leads to inaccurate detection.
Moreover, cuCatch's tags could potentially be tampered with as cuCatch directly
encodes the tag into the pointer yet does not check if a pointer arithmetic
overwrites it. GPUShield~\cite{lee2022shield} adopts a similar approach to
cuCatch and uses a customized hardware component to perform the bounds
check. Though the customized hardware reduces the runtime overhead, it also
prevents GPUShield from being deployed on commercial GPUs, where the hardware
is not modifiable. IMT~\cite{imt} is another hardware-based solution that uses
ECC metadata as tags. Similar to GPUShield, IMT also requires hardware
modification and therefore faces similar deployment challenges as GPUShield.
Therefore, we mark all the hardware-based solutions in \T~\ref{tbl:sum} as
undeployable. Moreover, IMT does not specifically target memory safety but
instead provides a general approach for pointer tagging.

\parh{Aligning-based.}~Aligning-based sanitizers take a different approach
compared to the above ones. Instead of performing bound checking at memory
access, they check pointers' bounds during arithmetic instructions. Such approaches enforce
the alignment of the allocated memory to $2^n$ and encode the alignment $n$ into
the pointers. During the pointer arithmetic, the sanitizer checks if the
resulting pointer is still within the $2^n$ bounds and terminate the program if it is
not. LMI~\cite{lee2025lmi} is a recent work that adopts this approach on GPUs.
LMI encodes the alignment information into the pointers and modifies the ALUs to restrict
pointer arithmetic. However, like other pointer aligning schemes, LMI
misses overflow that occurs within the $2^n$ alignment and introduces
substantial memory waste due to fragmentation. Furthermore, LMI has only
limited support for temporal memory corruption~(e.g., use-after-free); it marks
a pointer as invalid upon the free-site and thus cannot detect corruption if
the pointer is copied prior to the free operation. Lastly, similar to
GPUShield, LMI also requires modification of the GPU hardware, which limits its
deployment on commercial GPUs. Consequently, we mark its detectability of both
types of memory corruption as partial, and mark it as undeployable, in
\T~\ref{tbl:sum}.


\parh{Summary.}~\T~\ref{tbl:sum} \tool with current memory safety solutions on
GPUs. In this table, we can see that existing GPU memory safety solutions
either offer incomplete detection capability or can only be deployed with
modified hardware/proprietary software. \tool, on the other hand, is a
software-based solution that can be deployed on commercial NVIDIA GPUs.
Furthermore, by adopting a hybrid design of pointer tagging and in-band
metadata, \tool delivers comprehensive detection for both spatial and temporal
memory corruption.

\section{Motivation}
\label{sec:tm}

\parh{Characteristics of GPU workloads.}~Since GPUs are designed for
  highly parallel workloads, it is common for GPU applications to execute
  thousands of threads concurrently. This high degree of parallelism directly
  leads to a need for high-throughput memory access. However, current GPU
sanitizers, such as cuCatch, leverage shadow memory to store metadata, which
generates a considerable memory footprint and puts significant pressure on the
memory systems.

\parh{Deployment hurdles of prior work.}~A major problem for prior GPU
sanitizers is the difficulty of deployment. Currently, the only sanitizer
supported on commodity NVIDIA GPUs is the compute-sanitizer~\cite{compsan},
which, as shown in \T~\ref{tbl:sum}, has limited detection capability and
substantial performance overhead~(15$\times$ on average). Recent
studies such as cuCatch and LMI~\cite{tarek2023cucatch,lee2025lmi} either
depend on the proprietary NVIDIA toolchains or customized hardware, making them
inaccessible to academic researchers and community developers.

\parh{Insufficient detection capability.}~Existing work such as cuCatch often
achieves only partial detection of memory corruption. As shown in
\T~\ref{tbl:sum}, cuCatch is implemented at the compiler backend and cannot
obtain accurate information about allocation sizes from the frontend, leading to
incomplete detection. Moreover, cuCatch uses a heuristic scheme with limited
entropy~($2^{8}=256$) to detect temporal corruption, which misses corruption
when the number of allocated objects exceeds 256. Similar issues also exist with
LMI, which cannot handle temporal corruption with copied pointers.

\parh{\tool.}~These observations motivate us to design a new GPU
sanitizer -- \tool. We implement \tool using off-the-shelf features~(i.e., MMU)
of NVIDIA GPUs so that \tool can be directly deployed on commodity GPUs. We
also design a new hybrid metadata scheme that combines pointer-tagging and
in-band metadata to accurately detect both spatial and temporal 
corruption. Moreover, \tool takes the characteristics of GPU workloads into
account; its scheme leverages in-band metadata to minimize the pressure on the
cache and memory system, achieving minimum performance overhead.


\section{System Design}
\label{sec:design}

This section details the architecture of GPU-Fuzz. As illustrated in Figure~\ref{fig:arch}, the system is designed around a data flow that seamlessly connects constraint-based test case generation with cross-framework execution, monitoring, and log-based reproduction.

\begin{figure}[htbp]
	\centering
	\includegraphics[width=\linewidth]{figs/arch.pdf} % Note: You need to generate arch.pdf from draw/arch.mermaid
	\caption{The architecture of the GPU-Fuzz system.}
	\label{fig:arch}
\end{figure}

\subsection{Constraint Library and Operator Modeling}
The foundation of GPU-Fuzz is its ability to generate semantically valid inputs. This is achieved through a Constraint Library that models the operational semantics of deep learning operators. For each target operator (e.g., convolution, pooling, matrix multiplication), we define a set of constraints that capture the legitimate relationships between its parameters, such as tensor shapes, data types, kernel sizes, strides, and padding. These constraints are expressed using the Z3 SMT solver's Python API, creating a framework-agnostic representation of the operator's contract. This approach ensures that any test case generated by the solver is, by construction, guaranteed to pass the framework's initial validation checks.

\subsection{Constraint Solving and Test Case Generation}
The core of the fuzzing loop resides in the test case generator. The process begins by randomly selecting an operator template from the Constraint Library. The generator then populates the template with randomized, but plausible, target dimensions and parameters. These parameters, along with the operator's formal constraints, are passed to the Z3 solver. The solver's task is to find a concrete assignment of values that satisfies all constraints. The result is a set of valid parameters (e.g., \texttt{input\_shape}, \texttt{kernel\_size}, \texttt{stride}) and the corresponding output tensor shape. This entire configuration is saved as a JSON file, representing an abstract, executable test case.

\subsection{Cross-Framework Execution and Monitoring}
Once a set of valid parameters is generated, the test case is materialized into an executable script for a target framework (PyTorch, TensorFlow, or PaddlePaddle). This script creates the necessary tensors, populates them with random data, and invokes the operator on the GPU.

The execution is monitored by NVIDIA's \texttt{compute-sanitizer}. This tool can detect a wide range of GPU-level errors, such as out-of-bounds memory accesses and race conditions, that would not typically cause a visible crash at the Python API level. When a bug is detected, \texttt{compute-sanitizer} terminates the process and generates a detailed report, which serves as our primary bug oracle.

\subsection{Log-based Reproduction}
A critical component of GPU-Fuzz is its comprehensive logging system, designed to ensure bug reproducibility. For every test case, the fuzzer records a detailed JSON log containing:
\begin{itemize}
    \item The target operator and framework.
    \item The exact set of parameters used (e.g., kernel size, stride, padding).
    \item The shapes of all input tensors.
    \item The random seed used for data generation and any other stochastic choices.
\end{itemize}
In addition to the JSON log, the complete input tensors are saved to separate files. When a crash occurs, this combination of logs provides a complete, self-contained record of the conditions that triggered the bug. A developer can then use a simple script to load these artifacts and deterministically reproduce the failure, significantly streamlining the debugging process.
\section{Implementation}
\label{sec:impl}
Our implementation of GPU-Fuzz consists of approximately 2,000 lines of Python code. The system is orchestrated by a central controller that manages the fuzzing loop, which includes test case generation, execution, and logging.

\parh{Constraint Library and Generation.}
Our constraint library, implemented in \texttt{gen/ops.py}, currently provides models for 11 families of operators, including various types of convolutions, pooling, and element-wise arithmetic operations. Table~\ref{tab:operator_families} lists the supported operator families.

\begin{table*}[t]
\centering
\caption{Supported Operator Families and Specific Operators in GPU-Fuzz}
\label{tab:operator_families}
\begin{tabularx}{\textwidth}{lllX}
\toprule
\textbf{Category} & \textbf{Operator Family} & \textbf{Specific Operators} & \textbf{Notes / Examples} \\
\midrule
\multirow{2}{*}{Convolution} & \multirow{2}{*}{Convolution} & Conv (1d, 2d, 3d) & Covers 1D, 2D, and 3D standard convolutions. \\
& & ConvTranspose (1d, 2d, 3d) & Covers 1D, 2D, and 3D transposed convolutions. \\
\midrule
\multirow{6}{*}{Pooling} & \multirow{6}{*}{Pooling} & MaxPool (1d, 2d, 3d) & \\
& & AvgPool (1d, 2d, 3d) & \\
& & FractionalMaxPool (2d, 3d) & \\
& & LPPool & \\
& & AdaptiveAvgPool (1d, 2d, 3d) & \\
& & AdaptiveMaxPool (1d, 2d, 3d) & \\
\midrule
\multirow{5}{*}{Padding} & \multirow{5}{*}{Padding} & ConstantPad & \\
& & ReflectionPad & \\
& & ReplicationPad & \\
& & ZeroPad & \\
& & CircularPad & \\
\midrule
\multirow{2}{*}{Element-Wise} & Unary & (abs, sin, sqrt, ...) & Includes absolute value, sine, square root, etc. \\
& Binary & (add, sub, mul, ...) & Includes element-wise addition, subtraction, multiplication, etc. \\
\midrule
Matrix Ops & MatMul & MatMul / BMM & Standard and batched matrix multiplication. \\
\bottomrule
\end{tabularx}
\end{table*}

Building a model for a new operator family, such as \texttt{AbsConv}, typically requires 100-150 lines of code to define its parameters as Z3 symbolic variables and encode the mathematical relationships between them (e.g., the output size formula) as Z3 constraints. The \texttt{materialize} method in each operator class then iterates through valid solutions found by the Z3 solver.

\parh{Framework Instantiation.} The parameters generated by the Z3 solver are saved to a JSON file. The \texttt{model\_gen.py} script reads this file and materializes the abstract test case into code for a specific framework (PyTorch, TensorFlow, or PaddlePaddle). It uses the framework's native Python API (e.g., \texttt{torch.nn.Conv2d}) to construct the operator with the solved parameters, creates tensor objects with the specified shapes, populates them with random data, and executes the operation on the GPU.

\parh{Crash Detection with Compute Sanitizer.} A key technical challenge was reliably automating crash detection. Our main controller script, \texttt{controller.py}, uses the \texttt{pexpect} library to solve this. It spawns the test case execution script (\texttt{model\_gen.py}) as a child process, but wraps the entire command with NVIDIA's \texttt{compute-sanitizer}. The controller monitors the output for specific error patterns that \texttt{compute-sanitizer} emits upon detecting a GPU kernel error. When such a pattern is matched, it terminates the process and archives the logs for reproduction.



\begin{figure*}
	\centering
	\includegraphics[width=\textwidth]{./draw/draw.pdf}
	\caption{Execution slowdown of \tool on different benchmarks. Note that for LLM benchmarks, we measure the throughput~(tokens/s), which is higher is better; we measure other cases' execution time~(seconds), which is lower is better. PolyBench's \texttt{lu} and \texttt{gemm} cannot finish within 60 minutes under compute-sanitizer, we omit them in the figure.}
	\label{fig:runtime}
\end{figure*}

\section{Evaluation}
\label{sec:eval}

To evaluate \tool, we first benchmark its ability to detect different types of
GPU memory corruption in \S~\ref{subsec:sec_eval}. We then assess the
performance of \tool under various GPU workloads in \S~\ref{subsec:perf_eval}.

\parh{Evaluation setup.}~Our evaluation used the following software setup:
Linux 6.12.21, NVIDIA driver 570.144 and CUDA 12.8. The hardware setup
consists of an AMD Ryzen 9950X and an NVIDIA RTX 5090 with 32 GiB device memory.


\subsection{Security Evaluation}
\label{subsec:sec_eval}

\begin{table}[htbp]
	\caption{Security evaluation of \tool.}
	\label{tab:sec_eval}

	\centering
	\resizebox{\linewidth}{!}{
		\begin{threeparttable}
			\begin{tabular}{llcccccc}
				\hline
				\multicolumn{2}{c}{\textbf{Type}}     & \textbf{\#}     & \textbf{\begin{tabular}[c]{@{}c@{}}compute\\ sanitizer\end{tabular}} & \textbf{\begin{tabular}[c]{@{}c@{}}GPU\\ Shield\tnote{1}\end{tabular}} & \textbf{cuCatch\tnote{1}} & \textbf{LMI\tnote{1}} & \textbf{\tool}              \\ \hline
				\multirow{3}{*}{\textbf{Spatial}}     & \textbf{Global} & 4                                                                    & 2                                                                      & 2                         & 4                     & 2              & \textbf{4} \\ \cline{2-8}
				                                      & \textbf{Local}  & 8                                                                    & 0                                                                      & 6                         & 5                     & 6              & \textbf{8} \\ \cline{2-8}
				                                      & \textbf{Shared} & 3                                                                    & 1                                                                      & 0                         & 1                     & 1              & \textbf{3} \\ \hline
				\multicolumn{2}{c}{\textbf{Coverage}} & /               & 20\%                                                                 & 53.3\%                                                                 & 66.7\%                    & 60\%                  & \textbf{100\%}              \\ \hline
				\multirow{4}{*}{\textbf{Temporal}}    & \textbf{UAF}    & 8                                                                    & 1                                                                      & N.A.\tnote{2}             & 5                     & 6              & \textbf{8} \\ \cline{2-8}
				                                      & \textbf{UAS}    & 4                                                                    & 3                                                                      & N.A.                      & 4                     & 0              & \textbf{4} \\ \cline{2-8}
				                                      & \textbf{IF}     & 2                                                                    & 2                                                                      & N.A.                      & 2                     & 2              & \textbf{2} \\ \cline{2-8}
				                                      & \textbf{DF}     & 4                                                                    & 4                                                                      & N.A.                      & 4                     & 4              & \textbf{4} \\ \hline
				\multicolumn{2}{c}{\textbf{Coverage}} & /               & 55.6\%                                                               & /                                                                      & 83.3\%                    & 66.7\%                & \textbf{100\%}              \\ \hline
			\end{tabular}
			\begin{minipage}{0.9\linewidth}
				\small UAF: Use-After-Free, UAS: Use-After-Scope, IF: Invalid Free, DF: Double Free.
			\end{minipage}
			\begin{tablenotes}
				\item[1]Due to the lack of public implementations of GPUShield, cuCatch and LMI, we estimated their coverage based on the description in their papers.
				\item[2]GPUShield does not detect temporal memory corruption.
			\end{tablenotes}
		\end{threeparttable}
	}
\end{table}



We evaluated \tool using a benchmark similar to those described in the papers
of cuCatch and LMI, as neither project has released their security benchmarks.
Our benchmarks, which we constructed and extended based on their descriptions,
comprise a total of 33 programs: 15 contain spatial memory corruption and 18
contain temporal memory corruption. For spatial vulnerabilities, our benchmarks include
both adjacent and non-adjacent overflows. Additionally, for local
memory overflows, we evaluate both in-frame and cross-frame overflows. As for
temporal vulnerabilities, we include cases with both immediate and delayed
access. This diverse set of testcases provides a comprehensive evaluation of the
detectability of GPU sanitizers. Our benchmarks have been released at
\cite{cusan} to support future research. \T~\ref{tab:sec_eval} summarizes the
evaluation results of existing GPU sanitizers.

\parh{Spatial corruption.}~From the \F~\ref{tab:sec_eval}, we can observe that
compute-sanitizer detects only three of the 15 spatial memory corruption
programs. It misses all non-adjacent memory overflows and has limited coverage
for the adjacent ones. LMI and GPUShield demonstrate similar coverage as both
of them rely on a mechanism that encodes the $2^n$-aligned size into the
pointer\footnote{Unlike LMI, which relies on $2^n$-aligned size for detection,
GPUShield adopts it to improve the performance.}. This approach makes them
unable to detect memory overflows that occur within the $2^n$ range. In
contrast, \tool overcomes this limitation by embedding the exact size of a
buffer into the metadata, allowing it to accurately detect spatial overflow.
While cuCatch also captures the exact bounds of memory buffers, it fails to
infer the exact bounds of local/shared memory buffers. This is because the
bound information is not available in the compiler's backend, where cuCatch is
implemented. Implemented on the LLVM frontend, \tool can easily obtain the
exact bounds of memory buffers for all memory types, thereby avoiding the
limitations of cuCatch. Thus, \tool achieves the best coverage for spatial
corruption among all sanitizers.


\parh{Temporal corruption.}~As for temporal memory corruption,
compute-sanitizer detects 10 out of 18 cases; it misses all delayed UAF bugs
and one immediate UAS~(i.e., dereferencing a local pointer immediately after
the function returns). GPUShield does not offer protection against temporal
memory corruption and was excluded from the evaluation. cuCatch detects delayed
UAFs probabilistically by assigning one of 256 tags to each allocation.
However, when many buffers (>256) are allocated between the use-site and
free-site, tag collisions become likely. In our evaluation, cuCatch failed to
detect three delayed UAFs due to tag collision. LMI, on the other hand,
provides deterministic detection of temporal memory corruption via metadata
invalidation. Yet, since its metadata is embedded solely in the pointer, the
invalidation of the original pointer does not propagate to the pointers that
were copied before the free-site. As a result, LMI misses two UAFs and all UASs
where copied pointers are involved. In contrast, \tool uses VA-randomization to
detect delayed UAFs, which provides stronger heuristic protection~(i.e.,
$2^{41-A}$ as described in \S~\ref{subsec:temporal_meta}). This makes the tag
collision seen in cuCatch highly unlikely. Moreover, unlike LMI, \tool embeds
the metadata in both the pointer and the buffer, ensuring the
invalidation is visible to all use-site, including those of copied pointers.
Therefore, \tool achieves the best coverage in detecting temporal
corruption.


\subsection{Performance Evaluation}
\label{subsec:perf_eval}

We evaluated \tool's performance across a diverse set of 44 testcases,
summarized in \T~\ref{tab:perf_eval}. Our evaluation incorporates both
established GPU benchmark suites, such as Rodinia, Tango, and
PolyBench-GPU~\cite{polybenchgpu,tango,rodinia}, and two popular large language
models (LLMs) to assess performance on real-world workloads: LLaMA2 (7B) and
LLaMA3 (8B)~\cite{llama2,llama3}. All reported overheads are the average of
five iterations.


\begin{table}[htbp]
	\caption{Benchmarks used to evaluate \tool.}
	\label{tab:perf_eval}
	\centering
	\resizebox{\linewidth}{!}{
		\begin{tabular}{ccl}
			\hline
			\textbf{Suite} & \textbf{\#} & \multicolumn{1}{c}{\textbf{Testcases}}                                                                                                                                                                                                         \\ \hline
			Rodinia        & 17          & \begin{tabular}[c]{@{}l@{}}b+tree, bfs, backprop, lavaMD, bfs, gaussian\\ pathfinder, srad, particlefilter\\ lud, nn, particle\_naive, particle\_float\\ sradv1, sradv2,hotspot, hotspot3d\\ dwt2d, heartwall, needle, pathfinder\end{tabular} \\ \hline
			PolyBench      & 19          & \begin{tabular}[c]{@{}l@{}}conv2d, conv3d, adi, bicg, covar, \\ gramschmidt, jacobi1d, jacobi2d\\ fdtd, gemver, mvt, 2mm, 3mm\\ atax, corr, doitgen, gemm, gesummv, lu\end{tabular}                                                            \\ \hline
			Tango          & 6           & AlexNet, CifarNet, GRU, LSTM, ResNet, SqueezeNet                                                                                                                                                                                               \\ \hline
			LLM            & 2           & LLaMA2, LLaMA3                                                                                                                                                                                                                                 \\ \hline
		\end{tabular}
	}
\end{table}

\subsubsection{Runtime Overhead}
\parh{Execution slowdown.}~\F~\ref{fig:runtime} presents the performance
overhead of \tool and compute-sanitizer. On average, \tool introduces a 13\%
overhead to the execution time and an 11\% drop on the throughput of LLMs. In
contrast, the average overhead of compute-sanitizer is substantially higher; it
imposes a 15$\times$ overhead and reduces LLM throughput by 98\%. For worst
case scenarios, compute-sanitizer imposes a 153$\times$ overhead on the execution time
and a 98.5\% reduction on LLM throughput. \tool, on the other hand, incurs a
maximum overhead of 83\% on the execution time~(observed on \texttt{gemm}) and
an 18\% drop on LLaMA3's throughput. This maximum overhead of 83\% is observed
in the \texttt{gemm} testcase from PolyBench, which includes a naive
implementation of matrix multiplication with sparse memory access. We attribute
this overhead to its sparse memory access pattern, which has a negative impact
on the cache efficiency. On the same \texttt{gemm} testcase, compute-sanitizer
incurs an overhead of 153$\times$, which partially supports our analysis. We believe
this situation rarely happens in real-world applications with highly optimized
CUDA kernels~(e.g., LLaMAs)

Other related works, while valuable, are not directly comparable for practical
reasons. The software sanitizer cuCatch, for instance, reports a 19\% average
and 3.1$\times$ maximum overhead; as cuCatch's evaluation suites are not
open-source, we cite its published results\footnote{We contacted the authors of
cuCatch for this information, but have not received a response.}. In another
domain, hardware-assisted schemes like LMI show promising potential with a
minimal 0.2\% overhead. However, their effectiveness relies on specialized
hardware that is unavailable in current commercial GPUs, limiting their
immediate applicability. Our work, in contrast, focuses on a practical solution
designed for direct use and evaluation on today's hardware.

\parh{Memory overhead.}~Beyond execution time, memory consumption is another
critical metric for GPU sanitizers, with details summarized in
Table~\ref{tab:mem}. We omit compute-sanitizer as it is a closed-source tool
with no details on its design. LMI's memory overhead is a byproduct of its
requirement that all memory be $2^n$-aligned, meaning the specific overhead is
payload-dependent. cuCatch and \tool's memory overhead, on the other hand,
consists of a fixed part and a scalable part. cuCatch introduces a fixed
overhead of 160\,MiB from its initial metadata structure. In addition, it
imposes a scalable overhead of 12.5\%, requiring 32 bits of metadata for every
32 bytes of allocated memory. \tool allocates a fixed memory for the stack
epoch, which is only 16.5 MiB~(see \S~\ref{subsec:temporal_meta}). As for
scalable part, \tool only attaches an 8-byte in-band size for each
allocation.(see \S~\ref{subsec:spatial_meta}).
%
Though \tool enforces $2^n$-alignment on allocated buffers, this alignment is
applied only to their VAs. Unlike LMI, \tool does not allocate additional
physical memory to fill the padding region; the physical memory is mapped only
to the actually used portion. Consequently, this alignment introduces no extra
memory consumption. 

\begin{table}[htbp]
	\caption{Memory overhead of different GPU sanitizers.}
	\label{tab:mem}
	\centering
	\resizebox{.8\linewidth}{!}{
		\begin{tabular}{ll}
			\hline
			\textbf{Tool} & \textbf{Memory overhead}                       \\ \hline
			LMI           & $2^n$-aligned fragmentation                    \\ \hline
			cuCatch       & Fixed part: 160 MiB; Scale part: 12.5\%        \\ \hline
			\tool         & Fixed part: 16.5 MiB; Scale part: 8 bytes/alloc \\ \hline
		\end{tabular}
	}
\end{table}

Apart from the above analysis, we also measured the memory overhead of \tool
and other sanitizers. \T~\ref{tab:mem_overhead} summarizes the results. Since
the implementations of cuCatch and LMI are not available, we first recorded the
size of each allocation in the test program, and then calculated the overhead
according to the allocation profile and the designs described in their papers.
We omit compute-sanitizer in \T~\ref{tab:mem_overhead}, as there is neither its
design details nor an accurate way to track its memory usage (it does not use
\texttt{cudaMalloc}). Nonetheless, we estimated its memory usage using
\texttt{nvidia-smi}, observing an overhead ranging from 200\,MiB to 600\,MiB,
depending on the specific test program.


\begin{table}[htbp]
	\caption{Memory overhead of GPU sanitizers.}
	\label{tab:mem_overhead}
	\centering
	\resizebox{.7\linewidth}{!}{
		\begin{tabular}{rccc}
			\hline
			                                                                      & \textbf{cuCatch}                                                            & \textbf{LMI}                                                        & \textbf{\tool}                                                         \\ \hline
			\textbf{\begin{tabular}[c]{@{}r@{}}Max. Rel.\\ Overhead\end{tabular}} & \begin{tabular}[c]{@{}c@{}}5,121.5$\times$\\ \texttt{jacobi1d}\end{tabular} & \begin{tabular}[c]{@{}c@{}}2$\times$\\ \texttt{needle}\end{tabular} & \begin{tabular}[c]{@{}c@{}}528$\times$\\ \texttt{jacobi1d}\end{tabular} \\ \hline
			\textbf{\begin{tabular}[c]{@{}r@{}}Max. Abs.\\ Overhead\end{tabular}} & \begin{tabular}[c]{@{}c@{}}3.79 GiB\\ \texttt{llama2}\end{tabular}          & \begin{tabular}[c]{@{}c@{}}6.89 GiB\\ \texttt{llama2}\end{tabular}  & \begin{tabular}[c]{@{}c@{}}16.51 MiB\\ \texttt{llama3}\end{tabular}     \\ \hline
			\textbf{\begin{tabular}[c]{@{}r@{}}Average\\ Overhead\end{tabular}}   & \begin{tabular}[c]{@{}c@{}}0.73 GiB\\ 15\%\end{tabular}                     & \begin{tabular}[c]{@{}c@{}}1.10 GiB\\ 23\%\end{tabular}             & \begin{tabular}[c]{@{}c@{}}16.5 MiB\\ 0.3\%\end{tabular}               \\ \hline
		\end{tabular}
	}
\end{table}

\T~\ref{tab:mem_overhead} reports the maximum and average memory overhead for
each sanitizer. From this table, LMI demonstrates the best performance in terms
of relative overhead, which is 2$\times$, while cuCatch's and \tool's are
5,121.5$\times$ and 528$\times$, respectively. This superior relative
performance by LMI, however, is attributed to its $2^n$-aligned memory policy,
which inherently bounds its worst-case relative overhead at 2$\times$. The
\texttt{jacobi1d} is also a special case; it allocates a very small amount of
memory~(32\,KiB), allowing the fixed overheads of cuCatch~(160\,MiB) and
\tool~(16.5\,MiB) to dominate its relative overhead. Regarding absolute
overhead, both cuCatch and LMI incur significant memory overheads of 3.79\,GiB
and 6.89\,GiB on the LLM benchmark, whereas \tool only requires 16.51\,MiB of
additional memory. Concerning average overhead, LMI performs the worst,
incurring an average overhead of 23\% and 1.10\,GiB of additional memory.
cuCatch has a moderate average overhead of 15\% with 0.73\,GiB. \tool, on the
other hand, has a negligible average overhead of only 0.3\% and requires
merely 16.5\,MiB of additional memory. 

For LLMs, the memory overhead from cuCatch and LMI might trigger out-of-memory
errors, causing functional LLM applications to crash unexpectedly. In contrast,
\tool adds negligible memory overhead, making such errors highly unlikely. This
property is crucial for LLMs, where memory usage is substantial and sensitive
to even modest increases.

\subsubsection{Occupancy and Divergence}
Occupancy and divergence are two important concepts in GPU programming.
Occupancy refers to the number of active warps per SM, which depends on the
hardware resources used by each thread and reflects kernel parallelism. Low
occupancy indicates that the GPU is underutilized, potentially leading to
performance degradation. We used \texttt{nsight}~\cite{nsight} to measure the
occupancy of CUDA kernels before and after \tool's instrumentation. Across the
113 CUDA kernels analyzed (one GPU application may contain multiple kernels),
only 13 showed an occupancy drop of more than 10\% after instrumentation, six
of which involve matrix multiplication (e.g., \texttt{gemm}). The largest drop,
50\%, was observed in the \texttt{lavaMD} testcase from Rodinia. Notably,
occupancy changes did not directly correlate with \tool's runtime overhead:
\texttt{gemm} had the highest overhead of 83\% but only a 24\% occupancy drop,
while \texttt{lavaMD} had a negligible overhead (<1\%) but a 50\% drop. This
indicates that \tool's overhead primarily comes from the extra time spent on
instrumented instructions, rather than from reduced kernel occupancy.

Divergence refers to the situation where threads in a warp take different
execution paths on the same branch, causing the divergent instructions to be
executed serially, leading to slowdown. Though \tool inserts additional
conditional branches, these branches are solely for checking the validity of
the memory access. In other words, divergence in these checks indicates memory
corruption, and therefore the program should be terminated. Therefore, such
divergence is not expected in normal execution. To confirm this, we measured
the divergence rate before and after \tool's instrumentation, and observed no
measurable increase.


\subsubsection{Optimized Checks}
In this section, we evaluate the effectiveness of \tool's optimizations by
measuring both the number of checks removed and the resulting performance
improvement. As shown in \F~\ref{fig:opt}, applying the three optimization
rules described in \S~\ref{subsec:opt} eliminates, on average, near 20\% of checks
across the 44 GPU programs used in our performance evaluation. These results
demonstrate the effectiveness of the proposed rules in identifying and removing
redundant checks.

\begin{figure}[htbp]
	\centering
	\includegraphics[width=0.7\linewidth]{./draw/opt.pdf}
	\caption{Percentage of checks optimized by \tool.}
	\label{fig:opt}
\end{figure}

\parh{Performance gain.}~We also measured the performance improvement from
\tool's optimizations. On average, these optimizations reduce the execution
time by 3.5\% and improve LLM's throughput by nearly 2\%. The most significant
performance gain is observed in the \texttt{lud}, where \tool's optimizations
reduces the execution time by over 60\%. We attribute this to the large number
of shared memory accesses in \texttt{lud}; removing the redundant checks allows
the compiler to better optimize the shared memory access pattern, leading to
substantial performance improvement.

\section{Related Work}
\label{sec:rw}
In this section, we discuss sanitizers for CPU programs. We already reviewed
highly relevant GPU sanitizers in \S~\ref{subsec:first_rw}.
%
Extensive efforts have been devoted to developing efficient and effective
sanitizers for CPU memory safety. One of the most widely used sanitizers is
ASan~\cite{asan}. ASan allocates \textit{shadow memory} to track the memory
status and instruments the program to prevent invalid memory access. A major
drawback of this approach is a performance overhead of approximately
100\%~\cite{asan}. To reduce this overhead, various improvements have been
proposed. For example, ASan-{}-~\cite{asanmin} reduces ASan's overhead by
removing redundant checks. Specifically, ASan-{}- leverages static analysis to
identify and remove the redundant checks of ASan, achieving a performance
improvement of 30\% to 60\%. RSan~\cite{rsan} introduces an efficient redzone
scheme that can check ranges of memory to improve performance. Since ASan's
overhead primarily stems from instrumentation, researchers have also explored
sanitizer designs that rely on hardware
features~\cite{mtsan,dangzero,pacsan,floatzone}. MTSan~\cite{mtsan} utilizes
emerging memory tagging features in Arm to track memory status for
closed-source software. PACSan~\cite{pacsan} employs a similar hardware feature
named pointer authentication in Arm to tag pointers. Beyond memory tagging,
hardware acceleration has also been applied to common sanitizer operations. For
instance, \citet{floatzone} proposed FloatZone, which repurposes the
floating-point unit to accelerate bound checking.
DangZero~\cite{dangzero} achieves efficient UAF detection by delegating the
memory management to the sanitizer.

\section{Discussion}
\label{sec:disc}

\parh{Unsupported memory.}~Though \tool supports most of the memory types
in the NVIDIA GPUs~(i.e., local, global, and shared memory), it does not
support two rarely used memory APIs, dynamic shared memory and in-kernel
\texttt{malloc}. Dynamic shared memory allows the host to dynamically
allocate a part of the shared memory before kernel launch. Since its allocation
is exclusively managed by the NVIDIA's proprietary runtime via a unique kernel
launch syntax~(i.e.,
\texttt{<{}<{}<dim\textsubscript{grid},dim\textsubscript{blk},size\textsubscript{shared}>{}>{}>}),
\tool cannot intercept the allocation to associate metadata. In-kernel
\texttt{malloc} enables CUDA kernels to allocate global memory directly. Due to
similar challenges in intercepting such allocations, \tool does not support this feature. Fortunately, these two
features are rarely used in practice. Shared memory is very limited in
size~(for example, 64\,KiB) where static allocation is sufficient. As for
in-kernel \texttt{malloc}, it incurs a significant performance
penalty~\cite{kernelmalloc} compared to its host-side counterpart~(i.e.,
\texttt{cudaMalloc}), and is therefore seldom used.

\parh{Closed-source modules.}~Since \tool is designed as a LLVM pass in the
compiler's frontend, it cannot instrument closed-source modules. In the current
CUDA ecosystem, a significant portion of such modules comprises NVIDIA's
proprietary libraries, such as \texttt{cuDNN} and \texttt{cuBLAS}~\cite{cudnn}.
However, these libraries are highly optimized and extensively tested, making
the likelihood of memory bugs low. In addition, open-source alternatives exist,
such as \texttt{MAGMA} and \texttt{ArrayFire}~\cite{magma,arrayfire}.

\parh{Other platforms.} Though \tool is designed for NVIDIA GPUs, the underlying mechanism of \tool
does not depend on features specific to NVIDIA GPUs. Modern GPUs from other
vendors~(e.g., AMD, Intel and Apple) also support MMUs, making them viable
platforms for implementing \tool. However, the openness of GPU APIs varies
across vendors. For example, AMD provides an open-source CUDA-like API named
HIP~\cite{hip}, which could facilitate a swift port of \tool to AMD GPUs. In
contrast, Intel and Apple are more restrictive regarding their GPU APIs,
particularly low-level memory management APIs that \tool requires. This
presents challenges for porting \tool to those platforms. Nonetheless, since
low-level memory control is essential for high-performance computing, we expect
that other vendors will eventually expose such APIs.

\section{Conclusion}
\label{sec:conclusion}

\tool is the first practical GPU memory sanitizer that comprehensively detects
both spatial and temporal errors on commodity NVIDIA GPUs. It leverages only
off-the-shelf hardware and open toolchains. By combining pointer tagging with
in-band metadata, it ensures precise and low-overhead memory safety. Evaluated
on diverse benchmarks and LLMs, \tool achieves full detection coverage with
13\% runtime overhead and minimal memory use.

\appendix
\section*{Ethical Considerations}
This paper introduces a software-based memory safety solution for commodity
NVIDIA GPUs, aiming to improve system reliability and security without causing
harm. All experiments were conducted locally using synthetic data, eliminating
any risks to personal privacy or public services. Our approach involves no
attack vectors against existing CUDA programs or NVIDIA GPUs, ensuring no risk
to the CUDA ecosystem or NVIDIA's interests.
%
We carefully conducted fair and reproducible comparisons between \tool and
competing systems by using identical software and hardware configurations,
maintaining equalized baselines, and reporting results based on the average of
multiple runs.

These practices align with ethical principles of beneficence, respect for
persons, and justice, ensuring that our research advances knowledge while
safeguarding privacy, fairness, and the integrity of existing systems.

\section*{Open Science}
We confirm that the submitted paper follows the open science policy of USENIX
Security' 26. The source code of \tool~(LLVM pass, dynamic linking library, and
all benchmarks) is publicly available at an anonymous site~\cite{cusan}. We will
continue to maintain the source code for future research after the paper is
accepted.



\bibliographystyle{plainnat}
\bibliography{ref}

\appendix

\end{document}
\endinput
